\documentclass[margin=10pt,tikz]{standalone}
%\documentclass[x11names]{article}
\usepackage{tikz}
\usepackage{kotex}
\usepackage{enumitem}

\usetikzlibrary{shapes,arrows,chains}


\tikzset{
	base/.style={draw, on chain, on grid, align=center, minimum height=4ex},
	proc/.style={base, rectangle, text width=8em, fill=orange!30},
	ano/.style={base, rectangle, text width=10em, align=left},
	test/.style={base, diamond, aspect=2, text width=5em, fill=green!30},
	term/.style={proc, rounded corners, fill=red!30},
	io/.style={base, trapezium, trapezium left angle=70, trapezium right angle=110, fill=blue!30},
	% coord node style is used for placing corners of connecting lines
	coord/.style={coordinate, on chain, on grid, node distance=6mm and 25mm},
	% -------------------------------------------------
	% Connector line styles for different parts of the diagram
	norm/.style={->,>=stealth, draw},
	left sided/.style={
		draw=none,
		append after command={
			[shorten <= -0.5\pgflinewidth]
			([shift={(-1.5\pgflinewidth,-0.5\pgflinewidth)}]\tikzlastnode.north east)
			edge([shift={( 0.5\pgflinewidth,-0.5\pgflinewidth)}]\tikzlastnode.north west) 
			([shift={( 0.5\pgflinewidth,-0.5\pgflinewidth)}]\tikzlastnode.north west)
			edge([shift={( 0.5\pgflinewidth,+0.5\pgflinewidth)}]\tikzlastnode.south west)            
			([shift={( 0.5\pgflinewidth,+0.5\pgflinewidth)}]\tikzlastnode.south west)
			edge([shift={(-1.0\pgflinewidth,+0.5\pgflinewidth)}]\tikzlastnode.south east)
		}
	},
	right sided/.style={
		draw=none,
		append after command={
			[shorten <= -0.5\pgflinewidth]
			([shift={(-1.5\pgflinewidth,-0.5\pgflinewidth)}]\tikzlastnode.north west)
			edge([shift={( 0.5\pgflinewidth,-0.5\pgflinewidth)}]\tikzlastnode.north east) 
			([shift={( 0.5\pgflinewidth,-0.5\pgflinewidth)}]\tikzlastnode.north east)
			edge([shift={( 0.5\pgflinewidth,+0.5\pgflinewidth)}]\tikzlastnode.south east)            
			([shift={( 0.5\pgflinewidth,+0.5\pgflinewidth)}]\tikzlastnode.south east)
			edge([shift={(-1.0\pgflinewidth,+0.5\pgflinewidth)}]\tikzlastnode.south west)
		}
	}
}

\begin{document}
%[node distance=2cm]
\begin{tikzpicture}[%
	>=triangle 60,              % Nice arrows; your taste may be different
	start chain=going below,    % General flow is top-to-bottom
	node distance=6mm and 50mm, % Global setup of box spacing
	every join/.style={norm},   % Default linetype for connecting boxes
	]
\node [term]       (start)	{시작};
\node [io,   join] (io1)	{오염사고\\[1mm]시나리오 설정};
\node [proc, join] (p1)		{최대유출 사건 발생};
\node [proc, join] (p22)	{인근 권역장비\\[1mm] 2일 작업};
\node [proc, join] (p3)		{회수목표치 회수완료};
\node [io, join]   (io2)	{광역내\\[1mm] 장비수량 산정};
\node [io, join]			{권역내 지역별\\[1mm] 장비수량 산정};
\node [term, join] (end)	{끝};

\node [proc, left=of p22] (p21)	{사고구역 권역장비\\[1mm] 3일 작업};
\node [proc, right=of p22] (p23)	{원거리 권역장비\\[1mm] 1일 작업};
%


%
\node [coord, right=of io2] (c1)  {};
%
\node [ano, left sided, right=of io2, yshift=-1cm,] (a1) {
	\footnotesize \begin{enumerate} [leftmargin=*,topsep=0pt]
	\item 하루 8시간 작업시\\[1mm]3일 이내에 회수가능
	\end{enumerate}
};
%
%\path (t1.west) to node [near start, yshift=1em] {No} (c1);
%  \draw [norm] (t1.west) -- (c1) |- (p3);
%\path (t1.south) to node [near start, xshift=1em] {Yes} (ee);

\draw (io2.east) -- (a1.west);
\draw [norm] (p1) -| (p21);
\draw [norm] (p1) -| (p23);
\draw [norm] (p21) |- (p3);
\draw [norm] (p23) |- (p3);
%\draw (p1.north east) -- (a1.north west);
%\draw (p1.south east) -- (a1.south west);
%\draw (p2.north east) -- (a2.north west);
%\draw (p2.south east) -- (a2.south west);
\end{tikzpicture}

\begin{tikzpicture}[%
>=triangle 60,              % Nice arrows; your taste may be different
start chain=going below,    % General flow is top-to-bottom
node distance=6mm and 50mm, % Global setup of box spacing
every join/.style={norm},   % Default linetype for connecting boxes
]

\node [term]       (start)	{시작};
\node [io,   join] (io1)	{오염사고\\[1mm]시나리오 설정};
\node [proc, join] 			{무작위로 지역별로\\[1mm]장비 배치};
\node [proc, join] (p1)		{사건 발생};
\node [proc, join, very thick, style=double] (p22)		{인근 장비 항해\\[1mm] 및 동원시간을\\[1mm]고려하여 작업};
\node [proc, join] (p3)		{회수목표치\\[1mm]회수완료};
\node [test, join, very thick, style=double] (t1)		{\footnotesize 작업시간이\\단축되었는가?};
\node [term, join] (end)	{끝};


\node [proc, right=of t1, below of=p22, very thick, style=double] (p4)		{\footnotesize 최적화 알고리즘으로 \\[1mm] 지역별 장비 재배치};
\node [proc, left=of p22] (p21) {사고구역 지역장비\\3일 작업};



\node [coord, right=of t1] (c1)  {};

%
\node [ano, left sided, right=of p1, yshift=1.5cm, text width=12em] (a1) {
	\footnotesize 
	실제 상황에 가까운 조건 고려
	\begin{enumerate} [leftmargin=0.6cm]
	\item 출항준비시간
	\item 항해속도, 항해시간
	\item 육로이동시간
	\end{enumerate}
};

\node [ano, right sided, left=of t1, xshift=-1cm, yshift=1cm, text width=16em] (a2) {
\footnotesize
\begin{enumerate} [leftmargin=*, topsep=0pt]
\item 작업시간은 작업개시시각으로부터 3일이내\\(1일 작업시간 기준 08:00-16:00)
\item 작업시간 최소화
\end{enumerate}
};


\path (t1.south) to node [near start, xshift=1em] {Yes} (end);
\path (t1.east) to node [near start, yshift=1em] {No} (c1);
\draw (a1.west) -- (p22.east);
\draw (a2.east) -- (t1.north west);
\draw [norm] (t1.east) -| (p4);
\draw [norm] (p4) |- (p1);

\draw [norm] (p1) -| (p21);
\draw [norm] (p21) |- (p3);

\end{tikzpicture}

\end{document}
